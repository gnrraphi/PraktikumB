\documentclass[a4paper,10pt]{scrartcl}

\usepackage[utf8]{inputenc}
\usepackage[ngerman]{babel}
\usepackage[T1]{fontenc}
\usepackage{amsmath}
\usepackage[section]{placeins}
\usepackage{graphicx}
\usepackage{esvect}
\usepackage{amssymb}



\title{Praktikum B Vorbereitung zu Versuch "og"}
\author{Leon Machtl und Raphael Lehner}
\date{28.11.2019}

\begin{document}
	\maketitle
	\tableofcontents
	\newpage
	
	\section{Einleitung zum Versuch}
		In diesem Versuch wird mit einfachen, aber dennoch wichtigen Geräten experimentiert. Dabei steht die eigene Beobachtung im Vordergrund und die quantitative Auswertungen beruhen auf den Gesetzen der geometrischen Optik. Die geometrische Optik macht Gebrauch von dem sogenannten Strahlenmodell des Lichtes. Damit wird, vereinfacht ausgedrückt, der Weg des Lichtes auf Linien in rein geometrischer Weise betrachtet. So ein Linienstrahl entspricht natürlich in keiner Weise der Realität, weswegen er sich nicht experimentell realisieren lässt. Mit Hilfe der geometrischen Optik lässt sich aber dennoch die Funktion der Abbildung durch optische Geräte wie Lupen, Linsen, Brillen, Mikroskope, Fernrohre oder Teleskope häufig ausreichend genau beschreiben. Mathematisch kann man die geometrische Optik als Grenzfall für Wellenlängen, die gegen Null gehen, auffassen. Sie ist jedoch auch in dieser Betrachtungsweise nicht exakt. Effekte, die nicht mit Hilfe der geometrischen Optik beschrieben werden können, können häufig durch die Wellenoptik, beziehungsweise quantenmechanische Betrachtungen beschrieben werden. Dazu gehören beispielsweise Interferenz, Beugung, Polarisation, Absorption und Streuung des Lichtes. \\
		Im Rahmen dieses Experiments werden mehrere optische Geräte betrachtet. Bei diesen handelt es sich um die Lupe, mit der lediglich qualitative Beobachtungen durchgeführt werden sollen, das astronomische Fernrohr, das terrestrische Fernrohr, das holländische oder auf Galileische Fernrohr, sowie das Spiegelteleskop und der Diaprojektor. Dabei liegt der Fokus darin, die verwendeten Geräte selber zu bauen und ihre Wirkungsweise zu verstehen.\\
		Weitere Ziele des Versuchs sind theoretische Überlegungen und Berechnungen experimentell zu überprüfen, die Grundgesetze der geometrischen Objekt durch ihre Anwendung besser zu verstehen, Probleme der geometrischen Optik praktisch zu lösen, Versuchsaufbauten selbst zu justieren und zu optimieren und wie oben bereits erwähnt, wichtige optische Geräte und Instrumente selbst nachzubauen.
		
		\newpage
		
	\section{Aufgaben zur Vorbereitung}
		\subsection{Aufgabe 1}
			Wie lautet die Abbildungsgleichung für dünne Sammel- und Zerstreuungslinsen? Welche Näherungen
			werden bei der Herleitung gemacht? Was ist die Hauptebene einer Linse?\\
			\\
			(Hier Skizze mit Kenngrößen einer typischen Linse)\\
			\\
			In der Skizze entsprechen \(f\) und \(f'\) der sogenannten Brennweite, welche wiederum den Entfernungen der Brennpunkte \(F\) und \(F'\) von den beiden Hauptpunkten \(H\) und \(H'\) der Linse entspricht. Der Brennpunkt einer Linse bezeichnet den Punkt, an dem sich parallel zur Achse der Linse einfallende Strahlen treffen. Dabei sind \(f\) und \(f'\) verschieden, wenn die beiden Medien vor und hinter der Linse verschieden sind. Eine Entfernung zu einem Gegenstand \(G\) beziehungsweise die Entfernung zu seinem Bild \(G'\) mit jeweils der Höhe \(y\), beziehungsweise \(y'\) kann entweder von den Hauptpunkten oder den Brennpunkten aus gemessen werden. Dabei bezeichnen \(g\) und \(g'\) die Entfernung von den Hauptpunkten aus gemessen (Objektweite und Bildweite) und \(z\), beziehungsweise \(z'\) die Entfernung von den Brennpunkten. Betrachtet man nun eine dünne Linse, so fallen \(H\) und \(H'\) zusammen (Schnittpunkt Hauptebene und Linsenachse). Von der Bildseite einfallende parallele Strahlen, die auf eine Sammellinse fallen, werden in \(F'\) zusammengeführt. Bei einer Zerstreuunglinse ist dies nicht der Fall, \(F'\) liegt hier vor der Linse (Vergleiche Skizze).\\
			\\
			(Hier Skizze Situation dünne Sammel- und Zerstreuungslinse)\\
			\\
			Bei dünnen Linsen betrachtet man die Brechung näherungsweise in der Hauptebene und nicht auf der Oberfläche der Linse. Somit definiert der Brechungsindex der Linse nur eine und nicht zwei Brechungen. Diese Vereinfachung führt zu der Abbildungsgleichung für dünne Sammel- und Zerstreuunglinsen, welche durch
			\begin{align*}
			\frac{y'}{y}=\frac{g'}{g}
			\end{align*}
			ausgedrückt wird. Auch folgende Gleichung wird als Abbildungsgleichung bezeichnet:
			\begin{align*}
			\frac{1}{g}+\frac{1}{g'}=\frac{1}{f}
			\end{align*}
			
		\subsection{Aufgabe 2}
			Wodurch werden die Bildhelligkeit und das Gesichtsfeld beeinflusst bzw. begrenzt?\\
			\\
			Die Bildhelligkeit und das Gesichtsfeld werden durch Blenden beeinflusst, beziehungsweise begrenzt. Bei einem Auge übernimmt diese Aufgabe die Pupille.
			
		\subsection{Aufgabe 3}
			Leiten Sie die Formeln für die Vergrößerung von Lupe, den Fernrohren und dem Mikroskop
			her. Berücksichtigen Sie beim Fernrohr insbesondere die großen Gegenstandsweiten.\\
			\\
			Die Vergrößerung \(V\) eines optischen Gerätes ist definiert als
			\begin{align*}
			V=\frac{tan\epsilon}{tan\epsilon_{0}}
			\end{align*}	
			wobei \(\epsilon\) der Sehwinkel ist, unter dem das Objekt im optischen Instrument erscheint. \(\epsilon_{0}\) ist der Sehwinkel, unter dem man das Objekt ohne optische Hilfsmittel sieht. Der Winkel hängt vom Abstand des Auges zum Objekt ab und wir größer, je näher das Objekt ist. Optische Geräte, wie Lupe, Fernrohre und das Mikroskop dienen nun dazu, den Sehwinkel unter dem das Auge weit entfernte Objekte sieht, zu Vergrößern. Um die Winkelvergrößerung zu berechnen, wurde eine Bezugsweite von \(S=250mm\) definiert, bei der man Gegenstände mit bloßem Auge scharf sehen könnte.\\
			\\
			Lupe:\\
			\\
			Eine Lupe mit einer kurzen Brennweite ermöglicht die Betrachtung von kleinen Objekten mit Höhe \(y\) in der Entfernung \(s\) mit \(s<<S\) Der Abstand zwischen Auge und Objekt ist kleiner als \(2f\), wobei f wie oben der Brennweite entspricht. Die Situation ist in folgender Skizze ersichtlich:\\
			\\
			(Hier Skizze mit Lupe)\\
			\\
			Man kann ein virtuelles Bild mit Höhe \(y'\) beobachten. Befindet sich das Objekt genau in der Brennebene, so liegt \(y'\) im Unendlichen. In diesem Fall beträgt die Vergrößerung der Lupe
			\begin{align*}
			V_{L}=\frac{tan\epsilon}{tan\epsilon_{0}}=\frac{\frac{y}{f}}{\frac{y}{S}}=\frac{S}{f}
			\end{align*}
			\\
			Fernrohr:\\
			\\
			Bei einem astronomischen Fernrohr\\
			\\
			(Skizze Kepler Fernrohr)\\
			\\
			wird durch die Linse \(L_{1}\) von einem näherungsweise unendlich weit entfernten Objekt mit Höhe \(y\) ein umgekehrtes Bild der Höhe \(y'\) in der Brennebene erzeugt. Dabei handelt es sich um ein sogenanntes Zwischenbild, welches durch eine Linse \(L_{2}\) als virtuelles Bild im Unendlichen betrachtet wird, wenn \(F'_{1}=F_{2}\). In diesem Fall folgt für die Vergrößerung des astronomischen oder auch Keplerfernrohres
			\begin{align*}
			V_{kep}=\frac{tan\epsilon}{tan\epsilon_{0}}=\frac{\frac{y'_{1}}{f_{2}}}{\frac{y'_{1}}{f_{1}}}=\frac{f_{1}}{f_{2}}
			\end{align*}
			\\
			Bei einem Galilei Fernrohr\\
			\\
			(Skizze Galilei Fernrohr)\\
			\\
			\(L_{1}\) funktioniert wie beim astronomischen Fernrohr, jedoch wird vor der Brennebene bereits eine Zerstreuungslinse \(L_{2}\) mit negativer Brennweite \(f_{2}\) so platziert, dass \(F'_{1}\) und \(F_{2}\) zusammenfallen. Somit beobachtet man auch hier, wie schon beim astronomischen Fernrohr ein virtuelles Bild im Unendlichen, nun jedoch richtig herum. Somit ergibt sich für das Galilei-Fernrohr die selbe Vergrößerung wie schon für das astronomische Fernrohr, wenn ihre Beträge von \(f_{1}\) und \(f_{2}\) gleich sind:
			\begin{align*}
			V_{gal}=\frac{f_{1}}{|f_{2}|}
			\end{align*}
			\\
			Mikroskop:\\
			\\
			Mit einem Mikroskop werden kleine Objekte vergrößert betrachtet, dabei vergrößert es wieder den Sehwinkel unter dem ein Gegenstand dem Betrachter erscheint. Ein Mikroskop besteht aus zwei Linsensystemen, die das Objekt abbilden. Zuerst erzeugt ein Objektiv ein vergrößertes Zwischenbild, welches durch das Okular vergrößert betrachtet wird. Das hat zur Folge, dass der Betrachter ein vergrößertes virtuelles Bild wie bei einer Lupe erkennt. Dabei befindet sich das Objekt geringfügig außerhalb der Brennweite des Objektivs \(F_{1}\), wodurch ein reelles Zwischenbild erzeugt wird, welches erheblich größer, als der eigentliche beobachtete Gegenstand ist. Das Okular wird dann so angeordnet, dass das Zwischenbild innerhalb von \(f_{2}\) nahe von \(F_{2}\) liegt (siehe Skizze).\\
			\\
			(Hier Skizze Mikroskop)\\
			\\
			Somit wirkt das Okular als Lupe, wodurch das Zwischenbild als virtuelles Bild \(y''\) gesehen wird. Je nach Einstellung erscheint das Objekt dann im Unendlichen oder in \(S=250mm\). Somit ergibt sich für die Gesamtvergrößerung des Mikroskops durch das Produkt der Einzelvergrößerungen
			\begin{align*}
			V_{M}=\frac{y''}{S}\frac{S}{y}=\frac{y''}{y}
			\end{align*}
			
			\newpage
			
		\subsection{Aufgabe 4}
			In welche Entfernungsbereiche - bezogen auf die Brennweite der Lupe - können der zu betrachtende
			Gegenstand und das Auge gebracht werden?\\
			\\
			Dies ist in Aufgabe 3 unter dem Abschnitt der Lupe ersichtlich.
			
		\subsection{Aufgabe 5}
			Warum benutzt man in der Praxis meist Prismenfernrohre? Zeichnen Sie den Strahlengang!\\
			\\
			Man benutzt heutzutage in der Praxis zumeist Prismenfernrohre, da Fernrohre, wie das Galilei-Fernrohr nur eine relativ geringe Vergrößerung ermöglichen und außerdem relativ groß sein müssen. Zudem weisen sie Linsenfehler auf. Prismenfernrohre hingegen sind sehr platzsparend und daher praktischer konstruiert.\\
			\\
			(Skizze Prismenfernrohr)\\
			\\
			
		\subsection{Aufgabe 6}
			Erläutern Sie anhand einiger einfacher optischer Abbildungsanordnungen die Begriffe Apertur und
			Gesichtsfeldblende. Was will man mit ihnen bezwecken?\\
			\\
			Es ist sehr wichtig, die Strahlenbündelbegrenzung durch Blenden zu beachten, weil durch sie Bildhelligkeit, Abbildungsfehler, Auflösungsvermögen und Schärfentiefe der Abbildung beeinflusst und verändert werden können. Eine sogenannte Aperturblende beeinflusst die Helligkeit eines Bildes gleichmäßig durch Begrenzung der Apertur (Öffnungsweite) der optischen Geräte. Sie hat jedoch keinerlei Auswirkungen auf die Größe des Bildes. Ein Beispiel dafür ist die Linsenfassung (siehe Skizze). Wenn eine Aperturblende vor der Linse angebracht ist, dringen alle Strahlen durch die Blende in das System ein. Somit wird die Aperturblende wie ein Gegenstand abgebildet, wobei alle aus dem System austretenden Strahlen durch das virtuelle Bild der Aperturblende begrenzt werden. Häufig ist die Aperturblende zwischen zwei Linsen angebracht. Ein Beispiel für Aperturblenden aus der Natur stellt die Iris des menschlichen Auges dar.\\
			\\
			(Skizze Aperturblende)\\
			\\
			Eine Gesichtsfeldblende hingegen hat keine Auswirkung auf die Helligkeit eines Bildes, begrenzt jedoch den Bildausschnitt. Sie wird in der Bildebene plaziert oder in der Objektebene. Falls das Bild ein Diapositiv ist, ist die Dia-Maske die Feldblende, die den projizierten Bildausschnitt festlegt. Die Feldblende kann bei mehrstufiger Abbildung aber auch in der Ebene eines Zwischenbilds liegen. Das ist beim Mikroskop z.B. der Fall. \\
			\\
			(Skizze Gesichtsfeldblende)\\
			\\
			
		\subsection{Aufgabe 7}
			Wie lassen sich in optischen Systemen mit einem vorgegebenen Linsensatz kurze Brennweiten
			erzielen?\\
			\\
			Durch kleinere Blendendurchmesser und größere Blendenzahl lassen sich in optischen Systemen mit vorgegebenem Linsensatz kurze Brennweiten erzielen.
			
			
		
	\end{document}